\documentclass[a4paper]{article}

\usepackage[utf8]{inputenc}
\usepackage{url}

\newcommand{\strong}[1]{{\normalfont\fontseries{b}\selectfont #1}}
\newcommand{\code}[1]{\mbox{\texttt{#1}}}
\newcommand{\pkg}[1]{\strong{#1}}
\newcommand{\proglang}[1]{\textsf{#1}}

%% \VignetteIndexEntry{Introduction to the openNLP Package}

\usepackage{/usr/local/lib/R/share/texmf/Sweave}
\begin{document}
\title{Introduction to the \pkg{openNLP} Package}
\author{Ingo Feinerer}
\maketitle
\sloppy

\begin{abstract}
  The \pkg{openNLP} package.
\end{abstract}

\section*{Introduction}
The \pkg{openNLP} package provides a \proglang{R} interface to
\proglang{openNLP}\footnote{\url{http://opennlp.sourceforge.net/}}.

\section*{Loading the Package}
The package is loaded via
\begin{Schunk}
\begin{Sinput}
> library("openNLP")
\end{Sinput}
\end{Schunk}

\subsection*{Models}
It is highly recommended to use the \pkg{openNLPmodels} package to
provide the models necessary for full \pkg{openNLP} functionality.
\begin{Schunk}
\begin{Sinput}
> library("openNLPmodels")
\end{Sinput}
\end{Schunk}
This package provides default models both for English (\code{en}) and
Spanish (\code{es}).

\section*{Part-of-speech Tagging}
\begin{Schunk}
\begin{Sinput}
> sentence <- "This is a short sentence consisting of
+              some nouns, verbs, and adjectives."
> tagPOS(sentence, language = "en")
\end{Sinput}
\end{Schunk}
\begin{Schunk}
\begin{Soutput}
[1] "This/DT is/VBZ a/DT short/JJ sentence/NN consisting/VBG of/IN"
[2] "some/DT nouns,/JJ verbs,/NNS and/CC adjectives./VBG"          
\end{Soutput}
\end{Schunk}

\section*{Sentence Detection}
\begin{Schunk}
\begin{Sinput}
> s <- "This is a sentence. This another---but with dash-like
+       structures, and some commas. Maybe another with question
+       marks? Sure!"
> sentDetect(s, language = "en")
\end{Sinput}
\begin{Soutput}
[1] "This is a sentence. "                                                  
[2] "This another---but with dash-like\n      structures, and some commas. "
[3] "Maybe another with question\n      marks? "                            
[4] "Sure!"                                                                 
\end{Soutput}
\end{Schunk}

\section*{Tokenizer}
\begin{Schunk}
\begin{Sinput}
> s <- "¿Como se llama usted? El castellano es la lengua española oficial del Estado."
> tokenize(s, language = "es")
\end{Sinput}
\begin{Soutput}
 [1] "¿"          "Como"       "se"         "llama"      "usted"     
 [6] "?"          "El"         "castellano" "es"         "la"        
[11] "lengua"     "española"   "oficial"    "del"        "Estado"    
[16] "."         
\end{Soutput}
\end{Schunk}

\section*{Enhancements to tm}
The package provides transformations to enhance the \pkg{tm}
package. The functions \code{tmTagPOS}, \code{tmSentDetect}, and
\code{tmTokenize} are wrappers for above functions to be applied to
plain text documents.

\end{document}
